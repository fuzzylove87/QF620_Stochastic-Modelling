\documentclass{article}
\usepackage{amsmath}
\usepackage{amssymb}
\usepackage{parallel}
\usepackage[most]{tcolorbox}
\usepackage{amsthm}
\usepackage[a4paper,
top=1.5cm,
bottom=1.5cm,
left=1.8cm,
right=1.8cm,
heightrounded]
{geometry}
\renewcommand{\arraystretch}{2.5}

\title{QF 620 \protect \\ Stochastic Modelling in Finance\\
\textbf{Project Report}}

\date{}

\begin{document}
	
% PART 1: BLACK SCHOLES %%%%%%%%%%%%%%%%%%%%%%%%%%%%%%%%%%%%%%%%%%%%%%%%
	
\section{The Black-Scholes Model}
\begin{minipage}[t]{0.5\textwidth}
\begin{tcolorbox}[height=12.5cm,boxsep=5pt,arc=0pt,auto outer arc,colback=white,colframe=black]
\noindent \textbf{Solving SDE}\\ \\
\noindent Given that: $\boldsymbol{dS_t = r S_t dt + \sigma S_t dW_t}$\\ \\
\noindent Let: $X_t = \log (S_t) = f(S_t)$\\ \\
\noindent By Itô's Formula:
\begin{align*}
dX_t &= f'(S_t) dS_t + \frac{1}{2} f''(S_t) (dS_t)^2\\
&= \frac{1}{S_t} (r S_t dt + \sigma S_t dW_t) - \frac{1}{2} \frac{1}{S_t^2} (\sigma^2 S_t^2 dt)\\
&= \left(r-\frac{1}{2} \sigma^2 \right) dt + \sigma dW_t
\end{align*}
\noindent Integrating:
\begin{flalign*}
\int_{0}^{T} dX_t &= \left(r-\frac{1}{2} \sigma^2 \right) \int_{0}^{T} dt +  \sigma \int_{0}^{T} dW_t\\
\log \left( \frac{S_T}{S_0} \right) &= \left(r-\frac{1}{2} \sigma^2 \right)T + \sigma W_T\\
\boldsymbol{S_T }&\boldsymbol{= S_0 \exp\left[ \left(r-\frac{1}{2} \sigma^2 \right)T + \sigma W_T \right]}
\end{flalign*}
\noindent where $W_T \sim N(0,T) \sim \sqrt{T} N(0,1) \sim \sqrt{T} x$.
\end{tcolorbox}
\end{minipage}
\begin{minipage}[t]{0.5\textwidth}
\begin{tcolorbox}[height=12.5cm,boxsep=5pt,arc=0pt,auto outer arc,colback=white,colframe=black]
\noindent \textbf{Finding $\boldsymbol{x^*}$ where $\boldsymbol{S_T=K}$}
\begin{flalign*}
S_T &= K\\
K &= S_0 \exp\left[ \left(r-\frac{1}{2} \sigma^2 \right)T + \sigma \sqrt{T} x \right]\\
\log \left( \frac{K}{S_0} \right) &= \left(r-\frac{1}{2} \sigma^2 \right)T + \sigma \sqrt{T} x\\
\boldsymbol{x^* }&\boldsymbol{= \frac{\log\left( \frac{K}{S_0} \right) - \left( r-\frac{\sigma^2}{2} \right)T}{\sigma \sqrt{T}}}
\end{flalign*}\\
\noindent \textbf{Additional Comments}\\ \\
In the Black Scholes option pricing model, the valuation of the underlying asset is valued using a risk-neutral probability framework where the asset's prices are expressed  with a risk-free bond used as the numeraire security. \\ \\ As a result, the relative price process is turned into a martingale as the asset price process' previous drift term $\mu$ is replaced by the risk-free rate $r$.\\ \\
With the stock price process (discounted by the risk-free bond) expressed as a martingale, arbitrage opportunities are also eliminated.
\end{tcolorbox}
\end{minipage} \\ \\
\noindent \textbf{Derivation for Vanilla Call ($\boldsymbol{V_0^c}$) and Put ($\boldsymbol{V_0^p}$) Option Valuation Formulae}
\begin{flalign*}
V_0^c&=e^{-rT}\mathbb{E}[(S_T-K)^+]\\
&=e^{-rT}\left[ \int_{x^*}^{\infty} \frac{1}{\sqrt{2\pi}}\left( S_0 e^{\left(r-\frac{\sigma^2}{2}\right)T+\sigma \sqrt{T} x} - K\right)e^{-\frac{x^2}{2}} dx \right]\\
&=e^{-rT}\left[ S_0 e^{\left(r-\frac{\sigma^2}{2}\right)T}  \int_{x^*}^{\infty} \frac{1}{\sqrt{2\pi}} e^{-\frac{x^2-2 \sigma \sqrt{T} x + \sigma^2 T - \sigma^2 T}{2}} dx - K \int_{x^*}^{\infty} \frac{1}{\sqrt{2\pi}} e^{-\frac{x^2}{2}} dx\right]\\
&=e^{-rT}\left[ S_0 e^{rT} \int_{x^*}^{\infty} \frac{1}{\sqrt{2\pi}} e^{-\frac{(x-\sigma \sqrt{T})^2}{2}} dx - K \int_{x^*}^{\infty} \frac{1}{\sqrt{2\pi}} e^{-\frac{x^2}{2}} dx \right]\\
&= e^{-rT}\left[ S_0 e^{rT} \Phi (-x^* + \sigma \sqrt{T}) - K \Phi (-x^*) \right]\\
\boldsymbol{V_0^c }&\boldsymbol{= \boldsymbol{S_0 \Phi \left( \frac{\log\left( \frac{S_0}{K} \right) + \left( r+\frac{\sigma^2}{2} \right)T}{\sigma \sqrt{T}} \right) - K e^{-rT} \Phi \left( \frac{\log\left( \frac{S_0}{K} \right) + \left( r-\frac{\sigma^2}{2} \right)T}{\sigma \sqrt{T}} \right)}}\\
\boldsymbol{V_0^p }&\boldsymbol{=K e^{-rT} \Phi \left( \frac{\log\left( \frac{K}{S_0} \right) - \left( r-\frac{\sigma^2}{2} \right)T}{\sigma \sqrt{T}} \right) - S_0 \Phi \left( \frac{\log\left( \frac{K}{S_0} \right) - \left( r+\frac{\sigma^2}{2} \right)T}{\sigma \sqrt{T}} \right)}
\end{flalign*}\\
\noindent \textbf{Valuation Formulae for Other Option Types}
\\
\begin{center}
	\begin{tabular}{|c|c|c|}
		\hline
		\textbf{Option Type}& \textbf{Call} & \textbf{Put}\\
		\hline
		Digital Cash-or-Nothing&
		$e^{-rT} \Phi \left( \frac{\log\left( \frac{S_0}{K} \right) + \left( r-\frac{\sigma^2}{2} \right)T}{\sigma \sqrt{T}} \right)$&
		$e^{-rT} \Phi \left( \frac{\log\left( \frac{K}{S_0} \right) - \left( r-\frac{\sigma^2}{2} \right)T}{\sigma \sqrt{T}} \right)$
		\\
		\hline
		Digital Asset-or-Nothing& 
		$S_0 \Phi \left( \frac{\log\left( \frac{S_0}{K} \right) + \left( r+\frac{\sigma^2}{2} \right)T}{\sigma \sqrt{T}} \right)$&
		$S_0 \Phi \left( \frac{\log\left( \frac{K}{S_0} \right) - \left( r+\frac{\sigma^2}{2} \right)T}{\sigma \sqrt{T}} \right)$
		\\
		\hline
	\end{tabular}
\end{center}

\newpage

% PART 1: BLACK76 LOGNORMAL %%%%%%%%%%%%%%%%%%%%%%%%%%%%%%%%%%%%%%%%%%%%

\section{The Black76 Lognormal Model}
\begin{minipage}[t]{0.5\textwidth}
\begin{tcolorbox}[height=12.4cm,boxsep=5pt,arc=0pt,auto outer arc,colback=white,colframe=black]
	\noindent \textbf{Solving SDE}\\ \\
	\noindent Given that: $\boldsymbol{dF_t = \sigma F_t dW_t}$\\ \\
	\noindent Let: $X_t = \log (F_t) = f(F_t)$\\ \\
	\noindent where $F_T = S_T e^{rT}$.\\ \\
	\noindent By Itô's Formula:
	\begin{align*}
	dX_t &= f'(F_t) dF_t + \frac{1}{2} f''(F_t) (dF_t)^2\\
	&= \frac{1}{F_t} (\sigma F_t dW_t) - \frac{1}{2} \frac{1}{F_t^2} (\sigma^2 F_t^2 dt)\\
	&= -\frac{1}{2} \sigma^2 dt + \sigma dW_t
	\end{align*}
	\noindent Integrating:
	\begin{flalign*}
	\int_{0}^{T} dX_t &= - \frac{1}{2} \sigma^2 \int_{0}^{T} dt +  \sigma \int_{0}^{T} dW_t\\
	\log \left( \frac{F_T}{F_0} \right) &= -\frac{1}{2} \sigma^2 T + \sigma W_T\\
	\boldsymbol{F_T }&\boldsymbol{= F_0 \exp\left[ -\frac{1}{2} \sigma^2 T + \sigma W_T \right]}
	\end{flalign*}
	\noindent where $W_T \sim N(0,T) \sim \sqrt{T} N(0,1) \sim \sqrt{T} x$.
\end{tcolorbox}
\end{minipage}
\begin{minipage}[t]{0.5\textwidth}
\begin{tcolorbox}[height=12.4cm,boxsep=5pt,arc=0pt,auto outer arc,colback=white,colframe=black]
	\noindent \textbf{Finding $\boldsymbol{x^*}$ where $\boldsymbol{F_T=K}$}
	\begin{flalign*}
	F_T &= K\\
	K &= F_0 \exp\left[ -\frac{1}{2} \sigma^2 T + \sigma \sqrt{T} x \right]\\
	\log \left( \frac{K}{F_0} \right) &= -\frac{1}{2} \sigma^2 T + \sigma \sqrt{T} x\\
	\boldsymbol{x^* }&\boldsymbol{= \frac{\log\left( \frac{K}{F_0} \right) + \frac{\sigma^2 T}{2}}{\sigma \sqrt{T}}}
	\end{flalign*}\\
	\noindent \textbf{Additional Comments}\\ \\
	The Black76 Lognormal model is structurally identical to the Black-Scholes model but with forward prices, as opposed to stock prices, used instead. An advantage of using forward prices instead of stock prices is that the SDE for forward prices is more compact, driftless and is therefore a martingale.\\ \\
	Also, when using the Black76 model, the interest rate $r$ is not required for the computation of expected future option values. All inputs being constant, the Black-Scholes model and the Black76 Lognormal model should suggest at the same option values.
\end{tcolorbox}
\end{minipage}\\ 

\noindent \textbf{Derivation for Vanilla Call ($\boldsymbol{V_0^c}$) and Put ($\boldsymbol{V_0^p}$) Option Valuation Formulae}
\begin{flalign*}
V_0^c&=e^{-rT}\mathbb{E}[(F_T-K)^+]\\
&=e^{-rT}\left[ \int_{x^*}^{\infty} \frac{1}{\sqrt{2\pi}}\left( F_0 e^{\left( -\frac{\sigma^2 T}{2} + \sigma \sqrt{T} x \right)} - K\right)e^{-\frac{x^2}{2}} dx \right]\\
&=e^{-rT}\left[ F_0 e^{-\frac{\sigma^2 T}{2}} \int_{x^*}^{\infty} \frac{1}{\sqrt{2\pi}} e^{-\frac{x^2-2 \sigma \sqrt{T} x + \sigma^2 T - \sigma^2 T}{2}} dx - K \int_{x^*}^{\infty} \frac{1}{\sqrt{2\pi}} e^{-\frac{x^2}{2}} dx\right]\\
&=e^{-rT}\left[ F_0 \int_{x^*}^{\infty} \frac{1}{\sqrt{2\pi}} e^{-\frac{(x-\sigma \sqrt{T})^2}{2}} dx - K \int_{x^*}^{\infty} \frac{1}{\sqrt{2\pi}} e^{-\frac{x^2}{2}} dx \right]\\
&= e^{-rT}\left[ F_0 \Phi (-x^* + \sigma \sqrt{T}) - K \Phi (-x^*) \right]\\
\boldsymbol{V_0^c }&\boldsymbol{= e^{-rT}\left[F_0 \Phi \left( \frac{\log\left( \frac{F_0}{K} \right) + \frac{\sigma^2 T}{2}}{\sigma \sqrt{T}} \right) - K \Phi \left( \frac{\log\left( \frac{F_0}{K} \right) - \frac{\sigma^2 T}{2}}{\sigma \sqrt{T}}  \right) \right]}\\
\boldsymbol{V_0^p }&\boldsymbol{=e^{-rT}\left[K \Phi \left(  \frac{\log\left( \frac{K}{F_0} \right) + \frac{\sigma^2 T}{2}}{\sigma \sqrt{T}} \right) - F_0 \Phi \left( \frac{\log\left( \frac{K}{F_0} \right) - \frac{\sigma^2 T}{2}}{\sigma \sqrt{T}} \right) \right]}
\end{flalign*}\\
\noindent \textbf{Valuation Formulae for Other Option Types}
\\
\begin{center}
	\begin{tabular}{|c|c|c|}
		\hline
		\textbf{Option Type}& \textbf{Call} & \textbf{Put}\\
		\hline
		Digital Cash-or-Nothing&
		$e^{-rT} \Phi \left( \frac{\log\left( \frac{F_0}{K} \right) - \frac{\sigma^2 T}{2}}{\sigma \sqrt{T}} \right)$&
		$e^{-rT} \Phi \left( \frac{\log\left( \frac{K}{F_0} \right) + \frac{\sigma^2 T}{2}}{\sigma \sqrt{T}} \right)$
		\\
		\hline
		Digital Asset-or-Nothing& 
		$e^{-rT} F_0 \Phi \left( \frac{\log\left( \frac{F_0}{K} \right) + \frac{\sigma^2 T}{2}}{\sigma \sqrt{T}} \right)$&
		$e^{-rT} F_0 \Phi \left( \frac{\log\left( \frac{K}{F_0} \right) - \frac{\sigma^2 T}{2}}{\sigma \sqrt{T}} \right)$
		\\
		\hline
	\end{tabular}
\end{center}

\newpage

% PART 1: BACHELIER & BLACK76 NORMAL %%%%%%%%%%%%%%%%%%%%%%%%%%%%%%%%%%%%%%%%%%

\section{The Bachelier \& the Black76 Normal Model}
\noindent Structurally, the Bachelier and the Black76 Normal models are very similar. Derivations will be shown side-by-side to highlight their structural similarities.\\ \\
\begin{minipage}[t]{0.5\textwidth}
\begin{tcolorbox}[height=10.1cm,boxsep=5pt,arc=0pt,auto outer arc,colback=white,colframe=black]
	\noindent \textbf{The Bachelier Model}\\ \\
	\noindent \textbf{Solving SDE}\\ \\
	\noindent Given that: $\boldsymbol{dS_t = \sigma S_t dW_t}$\\ \\
	\noindent Integrating:
	\begin{flalign*}
	\int_{0}^{T} dS_t &= \sigma S_0 \int_{0}^{T} dW_t\\
	S_T - S_0 &= \sigma S_0 W_T\\
	\boldsymbol{S_T }&\boldsymbol{= S_0(1 + \sigma W_T)}
	\end{flalign*}
	\noindent where $W_T \sim N(0,T) \sim \sqrt{T} N(0,1) \sim \sqrt{T} x$.\\ \\
	\noindent \textbf{Finding $\boldsymbol{x^*}$ where $\boldsymbol{S_T=K}$}
	\begin{flalign*}
	S_T &= K\\
	K &= S_0 + S_0 \sigma \sqrt{T} x\\
	\boldsymbol{x^* } & \boldsymbol{= \frac{K-S_0}{S_0 \sigma \sqrt{T}}}
	\end{flalign*}
\end{tcolorbox}
\end{minipage}
\begin{minipage}[t]{0.5\textwidth}
\begin{tcolorbox}[height=10.1cm,boxsep=5pt,arc=0pt,auto outer arc,colback=white,colframe=black]
	\noindent \textbf{The Black76 Normal Model}\\ \\
	\noindent \textbf{Solving SDE}\\ \\
	\noindent Given that: $\boldsymbol{dF_t = \sigma F_t dW_t}$\\ \\
	\noindent Integrating:
	\begin{flalign*}
	\int_{0}^{T} dF_t &= \sigma F_0 \int_{0}^{T} dW_t\\
	F_T - F_0 &= \sigma F_0 W_T\\
	\boldsymbol{F_T }&\boldsymbol{= F_0(1 + \sigma W_T)}
	\end{flalign*}
	\noindent where $W_T \sim N(0,T) \sim \sqrt{T} N(0,1) \sim \sqrt{T} x$.\\ \\
	\noindent \textbf{Finding $\boldsymbol{x^*}$ where $\boldsymbol{F_T=K}$}
	\begin{flalign*}
	F_T &= K\\
	K &= F_0 + F_0 \sigma \sqrt{T} x\\
	\boldsymbol{x^* } & \boldsymbol{= \frac{K-F_0}{F_0 \sigma \sqrt{T}}}
	\end{flalign*}
\end{tcolorbox}
\end{minipage}\\ 

\noindent \textbf{Derivation for Vanilla Call ($\boldsymbol{V_0^c}$) and Put ($\boldsymbol{V_0^p}$) Option Valuation Formulae for Bachelier Model}\\ \\
\noindent Given the similarity in structure between the Bachelier and the Black76 Normal models, the option valuation formulae for the Black76 Normal model can be obtained simply by replacing the $\boldsymbol{S_T}$ and $\boldsymbol{S_0}$ terms in the Bachelier option valuation formulae with $\boldsymbol{F_T}$ and $\boldsymbol{F_0}$ respectively.
\begin{flalign*}
V_0^c&=e^{-rT}\mathbb{E}[(S_T-K)^+]\\
&=e^{-rT}\left[ \int_{x^*}^{\infty} \frac{1}{\sqrt{2\pi}}\left( S_0 + S_0 \sigma \sqrt{T} x - K\right) e^{-\frac{x^2}{2}} dx \right]\\
&=e^{-rT}\left[(S_0-K)\int_{x^*}^{\infty} \frac{1}{\sqrt{2\pi}} e^{-\frac{x^2}{2}} dx + S_0 \sigma \sqrt{T} \int_{x^*}^{\infty} x \cdot e^{-\frac{x^2}{2}} dx \right]\\
&=e^{-rT}\left[ (S_0-K)\int_{x^*}^{\infty} \frac{1}{\sqrt{2\pi}} e^{-\frac{x^2}{2}} dx - S_0 \sigma \sqrt{T} \int_{x^*}^{\infty} e^u du\right]\\
&=e^{-rT}\left[ (S_0-K)\int_{x^*}^{\infty} \frac{1}{\sqrt{2\pi}} e^{-\frac{x^2}{2}} dx - S_0 \sigma \sqrt{T} \left[ e^{-\frac{x^2}{2}}\right]^{\infty}_{x^*} \right]\\
&=e^{-rT}\left[ (S_0-K) \Phi (-x^*) + S_0 \sigma \sqrt{T} \phi (-x^*) \right]\\
\boldsymbol{V_0^c} &=\boldsymbol{e^{-rT}\left[ (S_0-K) \Phi \left(\frac{S_0-K}{S_0 \sigma \sqrt{T}}\right) + S_0 \sigma \sqrt{T} \phi \left(\frac{S_0-K}{S_0 \sigma \sqrt{T}}\right) \right]}\\
\boldsymbol{V_0^p} &=\boldsymbol{e^{-rT}\left[ (K-S_0) \Phi \left(\frac{K-S_0}{S_0 \sigma \sqrt{T}}\right) + S_0 \sigma \sqrt{T} \phi \left(\frac{K-S_0}{S_0 \sigma \sqrt{T}}\right) \right]}
\end{flalign*}\\
\noindent \textbf{Valuation Formulae for Other Option Types}
\\
\begin{center}
	\begin{tabular}{|c|c|c|}
		\hline
		\textbf{Option Type}& \textbf{Call} & \textbf{Put}\\
		\hline
		Digital Cash-or-Nothing&
		$e^{-rT} \Phi \left( \frac{S_0-K}{S_0 \sigma \sqrt{T}} \right)$&
		$e^{-rT} \Phi \left( \frac{K-S_0}{S_0 \sigma \sqrt{T}} \right)$
		\\
		\hline
		Digital Asset-or-Nothing& 
		$e^{-rT} S_0 \left[\Phi \left( \frac{S_0-K}{S_0 \sigma \sqrt{T}} \right) + \sigma \sqrt{T} \phi \left( \frac{S_0-K}{S_0 \sigma \sqrt{T}} \right)\right]$&
		$e^{-rT} S_0 \left[\Phi \left( \frac{K-S_0}{S_0 \sigma \sqrt{T}} \right) - \sigma \sqrt{T} \phi \left( \frac{K-S_0}{S_0 \sigma \sqrt{T}} \right)\right]$
		\\
		\hline
	\end{tabular}
\end{center}

\newpage

% PART 1: DISPLACED DIFFUSION %%%%%%%%%%%%%%%%%%%%%%%%%%%%%%%%%%%%%%%%%%%%%%%%

\section{The Displaced Diffusion (DD) Model}
\begin{minipage}[t]{0.58\textwidth}
\begin{tcolorbox}[height=15.5cm,boxsep=5pt,arc=0pt,auto outer arc,colback=white,colframe=black]
	\noindent \textbf{Solving SDE}\\ \\
	\noindent Given that: $\boldsymbol{dF_t=\sigma [ \beta F_t + (1 - \beta) F_0 ] dW_t}$\\ \\
	\noindent Let: $X_t = \log (\beta F_t + (1- \beta) F_0) = f(F_t)$\\ \\
	\noindent By Itô's formula: 
	\begin{flalign*}
	dX_t&=f'(F_t)dF_t + \frac{1}{2} f''(F_t)(dF_t)^2\\
	&=\frac{\beta}{\beta F_t + (1- \beta) F_0}(\sigma (\beta F_t + (1 - \beta) F_0) dW_t)\\
	& - \frac{1}{2}\frac{\beta^2}{(\beta F_t + (1- \beta) F_0)^2}(\sigma^2 (\beta F_t + (1 - \beta) F_0)^2 d_t)\\
	&= \beta \sigma dW_t - \frac{1}{2} \beta^2 \sigma^2 dt
	\end{flalign*}
	\noindent Integrating:\\
	\begin{flalign*}
	\int_{0}^{T} dX_t &= \beta \sigma \int_{0}^{T} dW_t - \frac{1}{2} \beta^2 \sigma^2 \int_{0}^{T} dt\\
	X_T - X_0 &= \beta \sigma W_T - \frac{1}{2} \beta^2 \sigma^2 T\\
	\log \left( \frac{\beta F_T + (1 - \beta) F_0}{\beta F_0 + (1 - \beta) F_0} \right) &= \beta \sigma W_T - \frac{1}{2} \beta^2 \sigma^2 T\\
	\frac{\beta F_T + (1 - \beta) F_0}{F_0}&= e^{\beta \sigma W_T - \frac{1}{2} \beta^2 \sigma^2 T}\\
	\boldsymbol{F_T}&\boldsymbol{=\frac{F_0}{\beta}e^{\beta \sigma W_T -\frac{1}{2} \beta^2 \sigma^2 T} - \frac{1-\beta}{\beta} F_0}
	\end{flalign*}\\
	\noindent where $W_T \sim N(0,T) \sim \sqrt{T} N(0,1) \sim \sqrt{T} x$.
\end{tcolorbox}
\end{minipage}
\begin{minipage}[t]{0.42\textwidth}
\begin{tcolorbox}[height=15.5cm,boxsep=5pt,arc=0pt,auto outer arc,colback=white,colframe=black]
	\noindent \textbf{Black76 Lognormal Model: Parallels}\\ \\
	Drawing parallels between the Black76 (B76) lognormal model and the Displaced Diffusion (DD) model when $\beta = 1$, we can see that:
	\begin{flalign*}
	F_{0,\textnormal{B76}} &\to \frac{F_{0,\textnormal{DD}}}{\beta}\\ K_{\textnormal{B76}} &\to K_{\textnormal{DD}} + \frac{1 - \beta}{\beta} F_{0,\textnormal{DD}}\\
	\sigma_{\textnormal{B76}} &\to \beta \sigma_{\textnormal{DD}}\\
	T_{\textnormal{B76}} &\to T_{\textnormal{DD}}
	\end{flalign*}\\
	\noindent \textbf{As such, $\boldsymbol{x^*}$, where $\boldsymbol{F_T=K}$ is:}
	\begin{flalign*}
	x^* = \frac{\log\left( \frac{K + (1 - \beta)/(\beta) F_0}{F_{0} / \beta} \right) + \frac{(\beta \sigma)^2 T}{2}}{\beta \sigma \sqrt{T}}
	\end{flalign*}\\
	\textbf{Other Comments}\\ \\
	The DD model allows for option valuations using a "blend" of the Black76 Normal and Lognormal models via the manipulation of the Beta ($\beta$) variable. This is to allow for a better "fit" of calculated implied volatilities with the observed volatility "smile" seen in actual option markets. With that said, the DD model still falls short on many fronts, as will be shown in the later parts of this report.
\end{tcolorbox}
\end{minipage}\\

\noindent \textbf{Valuation Formulae for Options}
\\

%F -> F_0 / \beta
%F -> \frac{F_0}{\beta}
%K -> K + F_0((1 - \beta)/\beta)
%K -> \left( K + F_0\frac{1- \beta}{\beta} \right)
%sigma -> \beta \sigma
%sigma -> (\beta \sigma)

\begin{center}
	\begin{tabular}{|c|c|}
		\hline
		\textbf{Option Type}& \textbf{Valuation Formula}\\
		\hline
		Vanilla Call&
		$e^{-rT}\left[\frac{F_0}{\beta} \Phi \left( \frac{\log\left( \frac{F_0 / \beta}{K + F_0((1 - \beta)/\beta)} \right) + \frac{(\beta \sigma)^2 T}{2}}{\beta \sigma \sqrt{T}} \right) - \left( K + F_0\frac{1- \beta}{\beta} \right) \Phi \left( \frac{\log\left( \frac{F_0 / \beta}{K + F_0((1 - \beta)/\beta)} \right) - \frac{(\beta \sigma)^2 T}{2}}{\beta \sigma \sqrt{T}}  \right) \right]$
		\\
		\hline
		Vanilla Put&
		$e^{-rT}\left[\left( K + F_0\frac{1- \beta}{\beta} \right) \Phi \left(  \frac{\log\left( \frac{K + F_0((1 - \beta)/\beta)}{F_0 / \beta} \right) + \frac{(\beta \sigma)^2 T}{2}}{\beta \sigma \sqrt{T}} \right) - \frac{F_0}{\beta} \Phi \left( \frac{\log\left( \frac{K + F_0((1 - \beta)/\beta)}{F_0 / \beta} \right) - \frac{(\beta \sigma)^2 T}{2}}{\beta \sigma \sqrt{T}} \right) \right]$
		\\
		\hline
		Digital CoN Call&
		$e^{-rT} \Phi \left( \frac{\log\left( \frac{F_0 / \beta}{K + F_0((1 - \beta)/\beta)} \right) - \frac{(\beta \sigma)^2 T}{2}}{\beta \sigma \sqrt{T}} \right)$
		\\
		\hline
		Digital CoN Put&
		$e^{-rT} \Phi \left( \frac{\log\left( \frac{K + F_0((1 - \beta)/\beta)}{F_0 / \beta} \right) + \frac{(\beta \sigma)^2 T}{2}}{\beta \sigma \sqrt{T}} \right)$
		\\
		\hline
		Digital AoN Call& 
		$e^{-rT} \frac{F_0}{\beta} \Phi \left( \frac{\log\left( \frac{F_0 / \beta}{K + F_0((1 - \beta)/\beta)} \right) + \frac{(\beta \sigma)^2 T}{2}}{\beta \sigma \sqrt{T}} \right)$
		\\
		\hline
		Digital AoN Put&
		$e^{-rT} \frac{F_0}{\beta} \Phi \left( \frac{\log\left( \frac{K + F_0((1 - \beta)/\beta)}{F_0 / \beta} \right) - \frac{(\beta \sigma)^2 T}{2}}{\beta \sigma \sqrt{T}} \right)$
		\\
		\hline
	\end{tabular}
\end{center}
\end{document}