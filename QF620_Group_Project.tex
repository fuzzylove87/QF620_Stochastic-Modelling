\documentclass{article}
\usepackage{amsmath}
\usepackage{amssymb}
\usepackage{parallel}
\usepackage[most]{tcolorbox}
\usepackage{amsthm}
\usepackage[a4paper,
top=1.5cm,
bottom=1.5cm,
left=1.8cm,
right=1.8cm,
heightrounded]
{geometry}
\renewcommand{\arraystretch}{2.5}

\title{\textbf{QF620 Group Project}}

\author{\textbf{Group Members:} \\ \textbf{Joel Pang, Johnny Quek,} \\ \textbf{Wang Boyu, Woon Tian Yong}}

\date{}

\begin{document}
	\maketitle
		
	\begin{abstract}
		\noindent \textbf{Part I - Analytical Option Formulae} \\
		
		\noindent In this section, the valuation formulae for six types of options were derived based on four different valuation models. The six option types pertain to vanilla calls \& puts, digital cash-or-nothing calls \& puts and digital asset-or-nothing calls \& puts. The four option valuation models employed in the derivation process are the Black-Scholes model, the Bachelier model, the Black76 model and the Displaced-Diffusion model. \\ \\
		\noindent For each option valuation model (with the exception of the Displaced-Diffusion model), the valuation formula derivation process for a vanilla call option beginning with the Stochastic Differential Equation (SDE) is shown, with the resulting valuation formulae for other option types displayed as well. Additional comments touching on the context and/or the relationship(s) between the models are included in the report as well.\\
		
		\noindent \textbf{Part II - Model Calibration}\\ \\
		\noindent We then proceeded to use the models in Part I to derive the implied volatilities of options based on observed market prices. We first calibrated the Displaced-Diffusion model in an attempt to fit the implied volatility curve as seen across the various strikes, but found no single Beta parameter that can provide a good fit. We then used the SABR model, and upon calibration, found that the parameters allowed us to fit the IV curve very well. We report the parameters and methodology in Part II. \\
		
		\noindent \textbf{Part III - Static Replication}\\ \\		
		In Part 3, we used the observed ATM sigma value to price various derivative contracts using the Black-Scholes as well as the Bachelier model. Having access to option prices across various strikes, and with the calibrated SABR giving us the implied volatilities for each strike, we are also able to use the static-replication method to price the contracts. We then report the these contract values under these models. \\
		
		\noindent \textbf{Part IV - Dynamic Hedging}\\ 
		
		\noindent In part IV, our motive was to investigate the results of discrete delta hedging between both sets of paths at N=21-time steps, and at N=84-time steps on a Black Scholes simulated price path. Therefore, we proceeded to run a total of 50,000 paths to simulate the underlying stock prices, in which we short an At-The-Money call option at time-step = 0. We then delta-hedged the option at each time step in order to remain delta-neutral throughout the both sets of paths at N=21-time steps, and at N=84-time steps. Finally, we recorded the simulated final P\&L results across 50,000 paths for both sets of paths, and compared the differences between the results.	
		
	\end{abstract} 
	
% PART 1: BLACK SCHOLES %%%%%%%%%%%%%%%%%%%%%%%%%%%%%%%%%%%%%%%%%%%%%%%%
\section*{Part I: Analytical Option Formulae}
\section{The Black-Scholes Model}
\begin{minipage}[t]{0.5\textwidth}
	\begin{tcolorbox}[height=11.5cm,boxsep=5pt,arc=0pt,auto outer arc,colback=white,colframe=black]
		\noindent \textbf{Solving SDE}\\ \\
		\noindent Given that: $\boldsymbol{dS_t = r S_t dt + \sigma S_t dW_t}$\\ \\
		\noindent Let: $X_t = \log (S_t) = f(S_t)$\\ \\
		\noindent By Itô's Formula:
		\begin{align*}
		dX_t &= f'(S_t) dS_t + \frac{1}{2} f''(S_t) (dS_t)^2\\
		&= \frac{1}{S_t} (r S_t dt + \sigma S_t dW_t) - \frac{1}{2} \frac{1}{S_t^2} (\sigma^2 S_t^2 dt)\\
		&= \left(r-\frac{1}{2} \sigma^2 \right) dt + \sigma dW_t
		\end{align*}
		\noindent Integrating:
		\begin{flalign*}
		\int_{0}^{T} dX_t &= \left(r-\frac{1}{2} \sigma^2 \right) \int_{0}^{T} dt +  \sigma \int_{0}^{T} dW_t\\
		\log \left( \frac{S_T}{S_0} \right) &= \left(r-\frac{1}{2} \sigma^2 \right)T + \sigma W_T\\
		\boldsymbol{S_T }&\boldsymbol{= S_0 \exp\left[ \left(r-\frac{1}{2} \sigma^2 \right)T + \sigma W_T \right]}
		\end{flalign*}
		\noindent where $W_T \sim N(0,T) \sim \sqrt{T} N(0,1) \sim \sqrt{T} x$.
	\end{tcolorbox}
\end{minipage}
\begin{minipage}[t]{0.5\textwidth}
	\begin{tcolorbox}[height=11.5cm,boxsep=5pt,arc=0pt,auto outer arc,colback=white,colframe=black]
		\noindent \textbf{Finding $\boldsymbol{x^*}$ where $\boldsymbol{S_T=K}$}
		\begin{flalign*}
		S_T &= K\\
		K &= S_0 \exp\left[ \left(r-\frac{1}{2} \sigma^2 \right)T + \sigma \sqrt{T} x \right]\\
		\log \left( \frac{K}{S_0} \right) &= \left(r-\frac{1}{2} \sigma^2 \right)T + \sigma \sqrt{T} x\\
		\boldsymbol{x^* }&\boldsymbol{= \frac{\log\left( \frac{K}{S_0} \right) - \left( r-\frac{\sigma^2}{2} \right)T}{\sigma \sqrt{T}}}
		\end{flalign*}
		\noindent \textbf{Additional Comments}\\ \\
		In the Black Scholes option pricing model, the underlying stock is valued using a risk-neutral probability framework when a risk-free bond used as the numeraire security. \\ \\ As a result, the relative price process $\frac{S_t}{B_t}$ is transformed to a martingale while the stock price process' deterministic drift term $\mu$ is replaced by the risk-free rate $r$. The transformation of $\frac{S_t}{B_t}$ to a martingale also suggests that no arbitrage opportunities should exist.
	\end{tcolorbox}
\end{minipage} \\ \\
\noindent \textbf{Derivation for Vanilla Call ($\boldsymbol{V_0^c}$) and Put ($\boldsymbol{V_0^p}$) Option Valuation Formulae}
\begin{flalign*}
V_0^c&=e^{-rT}\mathbb{E}[(S_T-K)^+]\\
&=e^{-rT}\left[ \int_{x^*}^{\infty} \frac{1}{\sqrt{2\pi}}\left( S_0 e^{\left(r-\frac{\sigma^2}{2}\right)T+\sigma \sqrt{T} x} - K\right)e^{-\frac{x^2}{2}} dx \right]\\
&=e^{-rT}\left[ S_0 e^{\left(r-\frac{\sigma^2}{2}\right)T}  \int_{x^*}^{\infty} \frac{1}{\sqrt{2\pi}} e^{-\frac{x^2-2 \sigma \sqrt{T} x + \sigma^2 T - \sigma^2 T}{2}} dx - K \int_{x^*}^{\infty} \frac{1}{\sqrt{2\pi}} e^{-\frac{x^2}{2}} dx\right]\\
&=e^{-rT}\left[ S_0 e^{rT} \int_{x^*}^{\infty} \frac{1}{\sqrt{2\pi}} e^{-\frac{(x-\sigma \sqrt{T})^2}{2}} dx - K \int_{x^*}^{\infty} \frac{1}{\sqrt{2\pi}} e^{-\frac{x^2}{2}} dx \right]\\
&= e^{-rT}\left[ S_0 e^{rT} \Phi (-x^* + \sigma \sqrt{T}) - K \Phi (-x^*) \right]\\
\boldsymbol{V_0^c }&\boldsymbol{= \boldsymbol{S_0 \Phi \left( \frac{\log\left( \frac{S_0}{K} \right) + \left( r+\frac{\sigma^2}{2} \right)T}{\sigma \sqrt{T}} \right) - K e^{-rT} \Phi \left( \frac{\log\left( \frac{S_0}{K} \right) + \left( r-\frac{\sigma^2}{2} \right)T}{\sigma \sqrt{T}} \right)}}\\
\boldsymbol{V_0^p }&\boldsymbol{=K e^{-rT} \Phi \left( \frac{\log\left( \frac{K}{S_0} \right) - \left( r-\frac{\sigma^2}{2} \right)T}{\sigma \sqrt{T}} \right) - S_0 \Phi \left( \frac{\log\left( \frac{K}{S_0} \right) - \left( r+\frac{\sigma^2}{2} \right)T}{\sigma \sqrt{T}} \right)}
\end{flalign*}\\
\noindent \textbf{Valuation Formulae for Other Option Types}
\\
\begin{center}
	\begin{tabular}{|c|c|c|}
		\hline
		\textbf{Option Type}& \textbf{Call} & \textbf{Put}\\
		\hline
		Digital Cash-or-Nothing&
		$e^{-rT} \Phi \left( \frac{\log\left( \frac{S_0}{K} \right) + \left( r-\frac{\sigma^2}{2} \right)T}{\sigma \sqrt{T}} \right)$&
		$e^{-rT} \Phi \left( \frac{\log\left( \frac{K}{S_0} \right) - \left( r-\frac{\sigma^2}{2} \right)T}{\sigma \sqrt{T}} \right)$
		\\
		\hline
		Digital Asset-or-Nothing& 
		$S_0 \Phi \left( \frac{\log\left( \frac{S_0}{K} \right) + \left( r+\frac{\sigma^2}{2} \right)T}{\sigma \sqrt{T}} \right)$&
		$S_0 \Phi \left( \frac{\log\left( \frac{K}{S_0} \right) - \left( r+\frac{\sigma^2}{2} \right)T}{\sigma \sqrt{T}} \right)$
		\\
		\hline
	\end{tabular}
\end{center}

\newpage

% PART 1: BLACK76 LOGNORMAL %%%%%%%%%%%%%%%%%%%%%%%%%%%%%%%%%%%%%%%%%%%%

\section{The Black76 Lognormal Model}
\begin{minipage}[t]{0.5\textwidth}
	\begin{tcolorbox}[height=12.4cm,boxsep=5pt,arc=0pt,auto outer arc,colback=white,colframe=black]
		\noindent \textbf{Solving SDE}\\ \\
		\noindent Given that: $\boldsymbol{dF_t = \sigma F_t dW_t}$\\ \\
		\noindent Let: $X_t = \log (F_t) = f(F_t)$\\ \\
		\noindent where $F_T = S_T e^{rT}$.\\ \\
		\noindent By Itô's Formula:
		\begin{align*}
		dX_t &= f'(F_t) dF_t + \frac{1}{2} f''(F_t) (dF_t)^2\\
		&= \frac{1}{F_t} (\sigma F_t dW_t) - \frac{1}{2} \frac{1}{F_t^2} (\sigma^2 F_t^2 dt)\\
		&= -\frac{1}{2} \sigma^2 dt + \sigma dW_t
		\end{align*}
		\noindent Integrating:
		\begin{flalign*}
		\int_{0}^{T} dX_t &= - \frac{1}{2} \sigma^2 \int_{0}^{T} dt +  \sigma \int_{0}^{T} dW_t\\
		\log \left( \frac{F_T}{F_0} \right) &= -\frac{1}{2} \sigma^2 T + \sigma W_T\\
		\boldsymbol{F_T }&\boldsymbol{= F_0 \exp\left[ -\frac{1}{2} \sigma^2 T + \sigma W_T \right]}
		\end{flalign*}
		\noindent where $W_T \sim N(0,T) \sim \sqrt{T} N(0,1) \sim \sqrt{T} x$.
	\end{tcolorbox}
\end{minipage}
\begin{minipage}[t]{0.5\textwidth}
	\begin{tcolorbox}[height=12.4cm,boxsep=5pt,arc=0pt,auto outer arc,colback=white,colframe=black]
		\noindent \textbf{Finding $\boldsymbol{x^*}$ where $\boldsymbol{F_T=K}$}
		\begin{flalign*}
		F_T &= K\\
		K &= F_0 \exp\left[ -\frac{1}{2} \sigma^2 T + \sigma \sqrt{T} x \right]\\
		\log \left( \frac{K}{F_0} \right) &= -\frac{1}{2} \sigma^2 T + \sigma \sqrt{T} x\\
		\boldsymbol{x^* }&\boldsymbol{= \frac{\log\left( \frac{K}{F_0} \right) + \frac{\sigma^2 T}{2}}{\sigma \sqrt{T}}}
		\end{flalign*}\\
		\noindent \textbf{Additional Comments}\\ \\
		The Black76 Lognormal model is structurally identical to the Black-Scholes model but with forward prices, as opposed to stock prices, used instead. An advantage of using forward prices instead of stock prices is that the SDE for forward prices is more compact, driftless and is therefore a martingale.\\ \\
		Also, when using the Black76 model, the interest rate $r$ is not required for the computation of expected future option values. All inputs being constant, the Black-Scholes model and the Black76 Lognormal model should suggest at the same option values.
	\end{tcolorbox}
\end{minipage}\\ 

\noindent \textbf{Derivation for Vanilla Call ($\boldsymbol{V_0^c}$) and Put ($\boldsymbol{V_0^p}$) Option Valuation Formulae}
\begin{flalign*}
V_0^c&=e^{-rT}\mathbb{E}[(F_T-K)^+]\\
&=e^{-rT}\left[ \int_{x^*}^{\infty} \frac{1}{\sqrt{2\pi}}\left( F_0 e^{\left( -\frac{\sigma^2 T}{2} + \sigma \sqrt{T} x \right)} - K\right)e^{-\frac{x^2}{2}} dx \right]\\
&=e^{-rT}\left[ F_0 e^{-\frac{\sigma^2 T}{2}} \int_{x^*}^{\infty} \frac{1}{\sqrt{2\pi}} e^{-\frac{x^2-2 \sigma \sqrt{T} x + \sigma^2 T - \sigma^2 T}{2}} dx - K \int_{x^*}^{\infty} \frac{1}{\sqrt{2\pi}} e^{-\frac{x^2}{2}} dx\right]\\
&=e^{-rT}\left[ F_0 \int_{x^*}^{\infty} \frac{1}{\sqrt{2\pi}} e^{-\frac{(x-\sigma \sqrt{T})^2}{2}} dx - K \int_{x^*}^{\infty} \frac{1}{\sqrt{2\pi}} e^{-\frac{x^2}{2}} dx \right]\\
&= e^{-rT}\left[ F_0 \Phi (-x^* + \sigma \sqrt{T}) - K \Phi (-x^*) \right]\\
\boldsymbol{V_0^c }&\boldsymbol{= e^{-rT}\left[F_0 \Phi \left( \frac{\log\left( \frac{F_0}{K} \right) + \frac{\sigma^2 T}{2}}{\sigma \sqrt{T}} \right) - K \Phi \left( \frac{\log\left( \frac{F_0}{K} \right) - \frac{\sigma^2 T}{2}}{\sigma \sqrt{T}}  \right) \right]}\\
\boldsymbol{V_0^p }&\boldsymbol{=e^{-rT}\left[K \Phi \left(  \frac{\log\left( \frac{K}{F_0} \right) + \frac{\sigma^2 T}{2}}{\sigma \sqrt{T}} \right) - F_0 \Phi \left( \frac{\log\left( \frac{K}{F_0} \right) - \frac{\sigma^2 T}{2}}{\sigma \sqrt{T}} \right) \right]}
\end{flalign*}\\
\noindent \textbf{Valuation Formulae for Other Option Types}
\\
\begin{center}
	\begin{tabular}{|c|c|c|}
		\hline
		\textbf{Option Type}& \textbf{Call} & \textbf{Put}\\
		\hline
		Digital Cash-or-Nothing&
		$e^{-rT} \Phi \left( \frac{\log\left( \frac{F_0}{K} \right) - \frac{\sigma^2 T}{2}}{\sigma \sqrt{T}} \right)$&
		$e^{-rT} \Phi \left( \frac{\log\left( \frac{K}{F_0} \right) + \frac{\sigma^2 T}{2}}{\sigma \sqrt{T}} \right)$
		\\
		\hline
		Digital Asset-or-Nothing& 
		$e^{-rT} F_0 \Phi \left( \frac{\log\left( \frac{F_0}{K} \right) + \frac{\sigma^2 T}{2}}{\sigma \sqrt{T}} \right)$&
		$e^{-rT} F_0 \Phi \left( \frac{\log\left( \frac{K}{F_0} \right) - \frac{\sigma^2 T}{2}}{\sigma \sqrt{T}} \right)$
		\\
		\hline
	\end{tabular}
\end{center}

\newpage

% PART 1: BACHELIER & BLACK76 NORMAL %%%%%%%%%%%%%%%%%%%%%%%%%%%%%%%%%%%%%%%%%%

\section{The Bachelier \& the Black76 Normal Model}
\noindent Structurally, the Bachelier and the Black76 Normal models are very similar. Derivations will be shown side-by-side to highlight their structural similarities.\\ \\
\begin{minipage}[t]{0.5\textwidth}
	\begin{tcolorbox}[height=10.1cm,boxsep=5pt,arc=0pt,auto outer arc,colback=white,colframe=black]
		\noindent \textbf{The Bachelier Model}\\ \\
		\noindent \textbf{Solving SDE}\\ \\
		\noindent Given that: $\boldsymbol{dS_t = \sigma S_t dW_t}$\\ \\
		\noindent Integrating:
		\begin{flalign*}
		\int_{0}^{T} dS_t &= \sigma S_0 \int_{0}^{T} dW_t\\
		S_T - S_0 &= \sigma S_0 W_T\\
		\boldsymbol{S_T }&\boldsymbol{= S_0(1 + \sigma W_T)}
		\end{flalign*}
		\noindent where $W_T \sim N(0,T) \sim \sqrt{T} N(0,1) \sim \sqrt{T} x$.\\ \\
		\noindent \textbf{Finding $\boldsymbol{x^*}$ where $\boldsymbol{S_T=K}$}
		\begin{flalign*}
		S_T &= K\\
		K &= S_0 + S_0 \sigma \sqrt{T} x\\
		\boldsymbol{x^* } & \boldsymbol{= \frac{K-S_0}{S_0 \sigma \sqrt{T}}}
		\end{flalign*}
	\end{tcolorbox}
\end{minipage}
\begin{minipage}[t]{0.5\textwidth}
	\begin{tcolorbox}[height=10.1cm,boxsep=5pt,arc=0pt,auto outer arc,colback=white,colframe=black]
		\noindent \textbf{The Black76 Normal Model}\\ \\
		\noindent \textbf{Solving SDE}\\ \\
		\noindent Given that: $\boldsymbol{dF_t = \sigma F_t dW_t}$\\ \\
		\noindent Integrating:
		\begin{flalign*}
		\int_{0}^{T} dF_t &= \sigma F_0 \int_{0}^{T} dW_t\\
		F_T - F_0 &= \sigma F_0 W_T\\
		\boldsymbol{F_T }&\boldsymbol{= F_0(1 + \sigma W_T)}
		\end{flalign*}
		\noindent where $W_T \sim N(0,T) \sim \sqrt{T} N(0,1) \sim \sqrt{T} x$.\\ \\
		\noindent \textbf{Finding $\boldsymbol{x^*}$ where $\boldsymbol{F_T=K}$}
		\begin{flalign*}
		F_T &= K\\
		K &= F_0 + F_0 \sigma \sqrt{T} x\\
		\boldsymbol{x^* } & \boldsymbol{= \frac{K-F_0}{F_0 \sigma \sqrt{T}}}
		\end{flalign*}
	\end{tcolorbox}
\end{minipage}\\ 

\noindent \textbf{Derivation for Vanilla Call ($\boldsymbol{V_0^c}$) and Put ($\boldsymbol{V_0^p}$) Option Valuation Formulae for Bachelier Model}\\ \\
\noindent Given the similarity in structure between the Bachelier and the Black76 Normal models, the option valuation formulae for the Black76 Normal model can be obtained simply by replacing the $\boldsymbol{S_T}$ and $\boldsymbol{S_0}$ terms in the Bachelier option valuation formulae with $\boldsymbol{F_T}$ and $\boldsymbol{F_0}$ respectively.
\begin{flalign*}
V_0^c&=e^{-rT}\mathbb{E}[(S_T-K)^+]\\
&=e^{-rT}\left[ \int_{x^*}^{\infty} \frac{1}{\sqrt{2\pi}}\left( S_0 + S_0 \sigma \sqrt{T} x - K\right) e^{-\frac{x^2}{2}} dx \right]\\
&=e^{-rT}\left[(S_0-K)\int_{x^*}^{\infty} \frac{1}{\sqrt{2\pi}} e^{-\frac{x^2}{2}} dx + S_0 \sigma \sqrt{T} \int_{x^*}^{\infty} x \cdot e^{-\frac{x^2}{2}} dx \right]\\
&=e^{-rT}\left[ (S_0-K)\int_{x^*}^{\infty} \frac{1}{\sqrt{2\pi}} e^{-\frac{x^2}{2}} dx - S_0 \sigma \sqrt{T} \int_{x^*}^{\infty} e^u du\right]\\
&=e^{-rT}\left[ (S_0-K)\int_{x^*}^{\infty} \frac{1}{\sqrt{2\pi}} e^{-\frac{x^2}{2}} dx - S_0 \sigma \sqrt{T} \left[ e^{-\frac{x^2}{2}}\right]^{\infty}_{x^*} \right]\\
&=e^{-rT}\left[ (S_0-K) \Phi (-x^*) + S_0 \sigma \sqrt{T} \phi (-x^*) \right]\\
\boldsymbol{V_0^c} &=\boldsymbol{e^{-rT}\left[ (S_0-K) \Phi \left(\frac{S_0-K}{S_0 \sigma \sqrt{T}}\right) + S_0 \sigma \sqrt{T} \phi \left(\frac{S_0-K}{S_0 \sigma \sqrt{T}}\right) \right]}\\
\boldsymbol{V_0^p} &=\boldsymbol{e^{-rT}\left[ (K-S_0) \Phi \left(\frac{K-S_0}{S_0 \sigma \sqrt{T}}\right) + S_0 \sigma \sqrt{T} \phi \left(\frac{K-S_0}{S_0 \sigma \sqrt{T}}\right) \right]}
\end{flalign*}\\
\noindent \textbf{Valuation Formulae for Other Option Types}
\\
\begin{center}
	\begin{tabular}{|c|c|c|}
		\hline
		\textbf{Option Type}& \textbf{Call} & \textbf{Put}\\
		\hline
		Digital Cash-or-Nothing&
		$e^{-rT} \Phi \left( \frac{S_0-K}{S_0 \sigma \sqrt{T}} \right)$&
		$e^{-rT} \Phi \left( \frac{K-S_0}{S_0 \sigma \sqrt{T}} \right)$
		\\
		\hline
		Digital Asset-or-Nothing& 
		$e^{-rT} S_0 \left[\Phi \left( \frac{S_0-K}{S_0 \sigma \sqrt{T}} \right) + \sigma \sqrt{T} \phi \left( \frac{S_0-K}{S_0 \sigma \sqrt{T}} \right)\right]$&
		$e^{-rT} S_0 \left[\Phi \left( \frac{K-S_0}{S_0 \sigma \sqrt{T}} \right) - \sigma \sqrt{T} \phi \left( \frac{K-S_0}{S_0 \sigma \sqrt{T}} \right)\right]$
		\\
		\hline
	\end{tabular}
\end{center}

\newpage

% PART 1: DISPLACED DIFFUSION %%%%%%%%%%%%%%%%%%%%%%%%%%%%%%%%%%%%%%%%%%%%%%%%

\section{The Displaced Diffusion (DD) Model}
\begin{minipage}[t]{0.58\textwidth}
	\begin{tcolorbox}[height=15.5cm,boxsep=5pt,arc=0pt,auto outer arc,colback=white,colframe=black]
		\noindent \textbf{Solving SDE}\\ \\
		\noindent Given that: $\boldsymbol{dF_t=\sigma [ \beta F_t + (1 - \beta) F_0 ] dW_t}$\\ \\
		\noindent Let: $X_t = \log (\beta F_t + (1- \beta) F_0) = f(F_t)$\\ \\
		\noindent By Itô's formula: 
		\begin{flalign*}
		dX_t&=f'(F_t)dF_t + \frac{1}{2} f''(F_t)(dF_t)^2\\
		&=\frac{\beta}{\beta F_t + (1- \beta) F_0}(\sigma (\beta F_t + (1 - \beta) F_0) dW_t)\\
		& - \frac{1}{2}\frac{\beta^2}{(\beta F_t + (1- \beta) F_0)^2}(\sigma^2 (\beta F_t + (1 - \beta) F_0)^2 d_t)\\
		&= \beta \sigma dW_t - \frac{1}{2} \beta^2 \sigma^2 dt
		\end{flalign*}
		\noindent Integrating:\\
		\begin{flalign*}
		\int_{0}^{T} dX_t &= \beta \sigma \int_{0}^{T} dW_t - \frac{1}{2} \beta^2 \sigma^2 \int_{0}^{T} dt\\
		X_T - X_0 &= \beta \sigma W_T - \frac{1}{2} \beta^2 \sigma^2 T\\
		\log \left( \frac{\beta F_T + (1 - \beta) F_0}{\beta F_0 + (1 - \beta) F_0} \right) &= \beta \sigma W_T - \frac{1}{2} \beta^2 \sigma^2 T\\
		\frac{\beta F_T + (1 - \beta) F_0}{F_0}&= e^{\beta \sigma W_T - \frac{1}{2} \beta^2 \sigma^2 T}\\
		\boldsymbol{F_T}&\boldsymbol{=\frac{F_0}{\beta}e^{\beta \sigma W_T -\frac{1}{2} \beta^2 \sigma^2 T} - \frac{1-\beta}{\beta} F_0}
		\end{flalign*}\\
		\noindent where $W_T \sim N(0,T) \sim \sqrt{T} N(0,1) \sim \sqrt{T} x$.
	\end{tcolorbox}
\end{minipage}
\begin{minipage}[t]{0.42\textwidth}
	\begin{tcolorbox}[height=15.5cm,boxsep=5pt,arc=0pt,auto outer arc,colback=white,colframe=black]
		\noindent \textbf{Black76 Lognormal Model: Parallels}\\ \\
		Drawing parallels between the Black76 (B76) lognormal model and the Displaced Diffusion (DD) model when $\beta = 1$, we can see that:
		\begin{flalign*}
		F_{0,\textnormal{B76}} &\to \frac{F_{0,\textnormal{DD}}}{\beta}\\ K_{\textnormal{B76}} &\to K_{\textnormal{DD}} + \frac{1 - \beta}{\beta} F_{0,\textnormal{DD}}\\
		\sigma_{\textnormal{B76}} &\to \beta \sigma_{\textnormal{DD}}\\
		T_{\textnormal{B76}} &\to T_{\textnormal{DD}}
		\end{flalign*}\\
		\noindent \textbf{As such, $\boldsymbol{x^*}$, where $\boldsymbol{F_T=K}$ is:}
		\begin{flalign*}
		x^* = \frac{\log\left( \frac{K + (1 - \beta)/(\beta) F_0}{F_{0} / \beta} \right) + \frac{(\beta \sigma)^2 T}{2}}{\beta \sigma \sqrt{T}}
		\end{flalign*}\\
		\textbf{Other Comments}\\ \\
		The DD model allows for option valuations using a "blend" of the Black76 Normal and Lognormal models via the manipulation of the Beta ($\beta$) variable. This is to allow for a better "fit" of calculated implied volatilities with the observed volatility "smile" seen in actual option markets. With that said, the DD model still falls short on many fronts, as will be shown in the later parts of this report.
	\end{tcolorbox}
\end{minipage}\\

\noindent \textbf{Valuation Formulae for Options}
\\

%F -> F_0 / \beta
%F -> \frac{F_0}{\beta}
%K -> K + F_0((1 - \beta)/\beta)
%K -> \left( K + F_0\frac{1- \beta}{\beta} \right)
%sigma -> \beta \sigma
%sigma -> (\beta \sigma)

\begin{center}
	\begin{tabular}{|c|c|}
		\hline
		\textbf{Option Type}& \textbf{Valuation Formula}\\
		\hline
		Vanilla Call&
		$e^{-rT}\left[\frac{F_0}{\beta} \Phi \left( \frac{\log\left( \frac{F_0 / \beta}{K + F_0((1 - \beta)/\beta)} \right) + \frac{(\beta \sigma)^2 T}{2}}{\beta \sigma \sqrt{T}} \right) - \left( K + F_0\frac{1- \beta}{\beta} \right) \Phi \left( \frac{\log\left( \frac{F_0 / \beta}{K + F_0((1 - \beta)/\beta)} \right) - \frac{(\beta \sigma)^2 T}{2}}{\beta \sigma \sqrt{T}}  \right) \right]$
		\\
		\hline
		Vanilla Put&
		$e^{-rT}\left[\left( K + F_0\frac{1- \beta}{\beta} \right) \Phi \left(  \frac{\log\left( \frac{K + F_0((1 - \beta)/\beta)}{F_0 / \beta} \right) + \frac{(\beta \sigma)^2 T}{2}}{\beta \sigma \sqrt{T}} \right) - \frac{F_0}{\beta} \Phi \left( \frac{\log\left( \frac{K + F_0((1 - \beta)/\beta)}{F_0 / \beta} \right) - \frac{(\beta \sigma)^2 T}{2}}{\beta \sigma \sqrt{T}} \right) \right]$
		\\
		\hline
		Digital CoN Call&
		$e^{-rT} \Phi \left( \frac{\log\left( \frac{F_0 / \beta}{K + F_0((1 - \beta)/\beta)} \right) - \frac{(\beta \sigma)^2 T}{2}}{\beta \sigma \sqrt{T}} \right)$
		\\
		\hline
		Digital CoN Put&
		$e^{-rT} \Phi \left( \frac{\log\left( \frac{K + F_0((1 - \beta)/\beta)}{F_0 / \beta} \right) + \frac{(\beta \sigma)^2 T}{2}}{\beta \sigma \sqrt{T}} \right)$
		\\
		\hline
		Digital AoN Call& 
		$e^{-rT} \frac{F_0}{\beta} \Phi \left( \frac{\log\left( \frac{F_0 / \beta}{K + F_0((1 - \beta)/\beta)} \right) + \frac{(\beta \sigma)^2 T}{2}}{\beta \sigma \sqrt{T}} \right)$
		\\
		\hline
		Digital AoN Put&
		$e^{-rT} \frac{F_0}{\beta} \Phi \left( \frac{\log\left( \frac{K + F_0((1 - \beta)/\beta)}{F_0 / \beta} \right) - \frac{(\beta \sigma)^2 T}{2}}{\beta \sigma \sqrt{T}} \right)$
		\\
		\hline
	\end{tabular}
\end{center}
	
\section*{Part II: Model Calibration}
\setcounter{section}{0}
\section{Displaced-Diffusion Model}

\underline{Calibrated Results:}
\begin{itemize}
	\item Beta: 0.3658
\end{itemize}

\begin{figure}[ht]
	\centering
	\includegraphics[width= \linewidth]{DD.png}
	%\caption{Displaced Diffusion Model Calibration}
\end{figure}
\noindent\\
Implied volatility from market prices were extracted using the Displaced Diffusion (DD) model with Beta=1, equivalent to the Black-76 Lognormal option pricing model, and the resulting implied volatilities plotted as ``Market IV".\\
\noindent\\
Differing values of Betas were input into the DD model to estimate which Beta parameter would most closely represent the implied volatility observed in the market. The main challenge in fitting the model was that in attempting to fit the IV of the lower strikes, we have to accept a larger deviation in the IV of the higher strikes, and vice versa. A least-squares method was used to find the Beta that would most closely fit market IV. This resulted in Beta parameters of approximately 0.3658.\\
\noindent\\
We note that while it does a sufficiently good job at matching ATM strike IVs, it gives a poor fit for the IVs for strikes at the extreme ends. The reason for this is that the market is pricing in a much higher IV value for lower strikes, and much lower IV value for higher strikes than these models suggest. This could partially be explained by behavioural reasons rooted in risk aversion, i.e. investors are willing to pay relatively high premiums for protection against large downside moves. Investors are also less willing to pay higher premiums for much higher strikes, and this could be due to the skew of historical returns (i.e. absence of sharp spikes in upside returns). We thus need a model that is able to incorporate all these additional factors, and so we turn to the SABR model. \\


\section{SABR Model}
\underline{Calibrated Results:}

\begin{itemize}
	\item Alpha: 0.9908
	\item Rho: -0.2851
	\item Nu: 0.3522
\end{itemize}

\begin{figure}[ht]
	\centering
	\includegraphics[width= \linewidth]{SABR.png}
	%\caption{SABR Model Calibration}
\end{figure}
\noindent\\
We calibrated the SABR model using the least squares method and obtained the following parameters of Alpha, Rho, and Nu, while keeping Beta fixed at 0.8. The calibrated SABR model fits the market IV very well, and this can be explained by the extra parameters which allow us to tweak the shape of the curve.\\


\begin{figure}[ht]
	\centering
	\includegraphics[width= \linewidth]{SABR-All.png}
	%\caption{SABR Model Calibration}
\end{figure}

\noindent\\
\underline{Calibrated Results:}

\noindent\\SABR Alpha adjusts for the height of the curve, and represents the constant of the volatility parameter. It can be seen as a baseline IV that is built into all the option prices. \\
\noindent\\
In the Black-Scholes model, the volatility term is deterministic, whereas in the SABR model, volatility of returns are modelled as being correlated with stock returns. SABR Rho adjusts for the skew in the distribution of stock returns. Decreasing the Rho adjusts for a more negative skew in stock returns. This allows us to increase the IVs of the lower strikes (OTM puts) to a higher IV than what the Black-Scholes model would suggest, and decrease the IVs of the higher strikes (OTM calls). Given that the calibrated correlation parameter is slightly negative, it implies that volatility increases when stock prices decrease. The adjustment also accounts for the behavioural factor that investors are usually more worried about downside risks than upside risks, and thus are willing to pay relatively higher prices for OTM lower strike puts as compared to similarly distant higher strikes OTM calls. \\
\noindent\\
SABR Nu adjusts for the volatility of volatility, and is proportionate to the degree of kurtosis of stock returns. A higher value of Nu increases the IV (and thus prices) of both calls and puts. This, coupled with a negative Rho value, leads to a much better curve that fits market prices and its corresponding expectations. \\
	
\section*{Part III: Static Replication}	
\setcounter{section}{0}
\section{Payoff function  $S^3_T+2.5 \log{S_T}+10.0 $}
\begin{minipage}[t]{0.5\textwidth}
	\begin{tcolorbox}[height=9cm,boxsep=5pt,arc=0pt,auto outer arc,colback=white,colframe=black]
		\noindent \textbf{The Black-Scholes model}\\ \\
		\noindent $\boldsymbol{S_T }\boldsymbol{= S_0 \exp\left[ \left(r-\frac{1}{2} \sigma^2 \right)T + \sigma W_T \right]}$\\ 
		where $W_T \sim N(0,T) \sim \sqrt{T} N(0,1) \sim \sqrt{T} x$.\\
		\noindent Given that: $\boldsymbol{\mathbb{E}[e^{\theta W_t}]=e^{\frac{\theta ^2}{2}t}}$ 
		\begin{align*}
		\mathbb{E}(S^3_T) &=\mathbb{E}(S^3_0 e^{[3(r-\frac{\sigma^2}{2})t+3\sigma W_t]})\\
		&=S^3_0 e^{[3(r-\frac{\sigma^2}{2})t]} \mathbb{E} [e^{3\sigma W_t}]\\
		&=S^3_0 e^{[3(r-\frac{\sigma^2}{2})t]} e^{\frac{9 \sigma^{2} t}{2}}\\
		&=S^3_0 e^{3t(r+\sigma^{2})}
		\end{align*}
		\noindent Given that: $\mathbb{E}(W_t)=0$
		\begin{flalign*}
		\mathbb{E}(\log{S_t})&=\mathbb{E}(\log{S_0})+\mathbb{E}((r-\frac{\sigma^2}{2})T+\sigma W_t)\\
		&=\log{S_0}+(r-\frac{\sigma^2}{2})T
		\end{flalign*}
	\end{tcolorbox}
\end{minipage}
\begin{minipage}[t]{0.5\textwidth}
\begin{tcolorbox}[height=9cm,boxsep=5pt,arc=0pt,auto outer arc,colback=white,colframe=black]
		\noindent \textbf{Contract Value:}
		\begin{flalign*}
		V_0 &= e^{-rT}\mathbb{E}(S^3_T+2.5\log{S_T}+10.0)\\
		&=e^{-rT}[\mathbb{E}(S^3_T)+2.5\mathbb{E}(\log{S_T})+10.0]\\
		&=e^{-rT}\{S^3_0 e^{3T(r+\sigma^{2})}\\
		&+2.5[\log{S_0}+(r-\frac{\sigma^{2}}{2})T]+10.0\}\\
		&=810237819.0000
		\end{flalign*}\\
		\noindent \textbf{Additional Comments}\\ \\
		The underlying assumption of the Black=Scholes model is that implied volatilities are the same across all strikes. The At-The-Money option IV is used in the model, and so we use this value of sigma to value this contract.\\ \\
		 \\ \\
	\end{tcolorbox}
\end{minipage} \\ 
\begin{minipage}[t]{0.5\textwidth}
	\begin{tcolorbox}[height=9cm,boxsep=5pt,arc=0pt,auto outer arc,colback=white,colframe=black]
		\noindent \textbf{Bachelier Model}\\ \\
		$\boldsymbol{S_T} = S_0(1+\sigma W_T)$\\ 
		\noindent Given that $W_t\sim N(0,T)$, we can know that $\mathbb{E}(W_T)=0 \; , \;\mathbb{E}(W_T^2)=T,\textnormal{and} \mathbb{E}(W^3_T)=0$ 
		\begin{align*}
		\mathbb{E}(S^3_t)&=\mathbb{E}(S^3_0(1+\sigma W_t)^3)\\
		&=S^3_0\mathbb{E}(1+3\sigma W_t+3\sigma^2 W_t^2+\sigma^3 W_t^3)\\
		&=S^3_0(1+3\sigma t)
		\end{align*}
		\noindent And we need to calculate the $\mathbb{E}(\log{S_T})$ by numerical method, since:
		\begin{flalign*}
		&\mathbb{E}(\log{S_T}) = \mathbb{E}(\log{S_0}+\log{1+\sigma W_T})\\
		&=\log{S_0}+\frac{1}{\sqrt{2\pi}}\int_{-\infty}^{+\infty}{\log{(1+\sigma \sqrt{T}x)}e^{-\frac{x^2}{2}}dx}
		\end{flalign*}
	\end{tcolorbox}
\end{minipage}
\begin{minipage}[t]{0.5\textwidth}
	\begin{tcolorbox}[height=9cm,boxsep=5pt,arc=0pt,auto outer arc,colback=white,colframe=black]
		\noindent \textbf{Contract Value:}
		\begin{flalign*}
		V_0 &= e^{-rT}\mathbb{E}(S^3_T+\log{S_T}+10.0)\\
		&= e^{-rT}[S^3_0(1+3\sigma t)+\mathbb{E}(\log{S_T})+10.0]\\
		&=771273909.1027
		\end{flalign*}\\
		\noindent \textbf{Additional Comments}\\ \\
		The Bachelier model also uses a constant sigma, and so we use the same ATM option IV to value this contract. The At-The-Money option is most liquid amongst all strikes, and so it is the most representative.\\ \\
		 \\ \\
	\end{tcolorbox}
\end{minipage}\\ 
\begin{minipage}[t]{0.5\textwidth}
	\begin{tcolorbox}[height=5.5cm,boxsep=5pt,arc=0pt,auto outer arc,colback=white,colframe=black]
\subsection*{Static-replication}
\begin{flalign*}
&h(S_T)=S^3_T+2.5 \log{S_T}+10.0\\
&h'(S_T)=3S_T^2+\frac{2.5}{S_T}\\
&h''(S_T)=6S_T-\frac{2.5}{S_T^2}
\end{flalign*}
	\end{tcolorbox}
\end{minipage}
\begin{minipage}[t]{0.5\textwidth}
	\begin{tcolorbox}[height=5.5cm,boxsep=5pt,arc=0pt,auto outer arc,colback=white,colframe=black]
		\noindent \textbf{Contract Value:}
		\begin{align*}
		V_0 &= e^{-rT}h(F)+\int_{0}^{F} {h''(K)P(K)dK}\\
		&+ \int_{F}^{+\infty}{h''(K)C(K)dK}\\
		&=802756073.1462
\end{align*}
P(K) and C(K) are Black76 Lognormal option prices with strikes across K. SABR sigma is used as the sigma input. 
	\end{tcolorbox}
\end{minipage}

\section{``Model-free" integrated   variance  $\sigma^2_{\textnormal{MF}}T=\mathbb{E}[\int_{0}^{T} {\sigma^2_t dt}]$}
\begin{minipage}[t]{0.5\textwidth}
	\begin{tcolorbox}[height=8cm,boxsep=5pt,arc=0pt,auto outer arc,colback=white,colframe=black]
\subsection{Black-Scholes model}
Using the Black-Scholes model, we can derive  
\begin{align*}
\mathbb{E}[\int_{0}^{T}{\sigma^2_T dt}]&=2\mathbb{E}[\int_{0}^{T}\frac{dS_t}{S_t}]-2\mathbb{E}[\log{\frac{S_t}{S_0}}]\\
&=2rT-2\mathbb{E}[(r-\frac{1}{2} \sigma^2 )T + \sigma W_T]\\
&=\sigma^{2}T
\end{align*}
We should use the ATM option implied volatility:  $\sigma^2_{\textnormal{MF}}=0.2583$ \\ \\
	\end{tcolorbox}
\end{minipage}
\begin{minipage}[t]{0.5\textwidth}
	\begin{tcolorbox}[height=8cm,boxsep=5pt,arc=0pt,auto outer arc,colback=white,colframe=black]
\textbf{And given that}:
\begin{align*}
\mathbb{E}[\int_{0}^{T}{\sigma^2_T dt}]&=2e^{rT}\int_{0}^{F}{\frac{P(K)}{K^2}dK}\\
&+2e^{rT}\int_{F}^{+\infty}{\frac{C(K)}{K^2}dK}
\end{align*}
\subsection{Bachelier model}
Applying the Bachelier option pricer on P(K) and C(K), we obtain $\sigma^2_{\textnormal{MF}}=0.0920$ 

\subsection{Static-replication}
Using the Static-replication approach, the Black76 Lognormal P(K) and C(K) values, along with SABR sigma, we obtain $\sigma^2_{\textnormal{MF}}=0.0756$ 
	\end{tcolorbox}
\end{minipage}

\section*{Part IV: Delta Hedging}	
\setcounter{section}{0}
\section{Assumptions}
With the following assumptions of the options parameters below:\\ 

\begin{center}
	$S_0 = \$100 ,r = 0.05, \sigma = 0.2, K = \$100$ 
\end{center} 

\section{Results}

\noindent We generated a total of 50,000 paths across one month period to simulate discrete delta hedging and the following is a chart of the underlying asset's paths taken in the one period, in which the price varies from 80-130.\\ \\

\begin{center}
	\includegraphics[width=0.5\textwidth]{./images/BS_price_21.jpg}%
	\includegraphics[width=0.5\textwidth]{./images/BS_price_84.jpg}
\end{center}

\begin{center}
	\includegraphics[width=0.5\textwidth]{./images/error_21.jpg}%
	\includegraphics[width=0.5\textwidth]{./images/error_84.jpg}
\end{center}

\begin{center}
	\begin{tabular}{|c|c|}
		\hline
		\textbf{Mean P\&L}& \textbf{Standard Deviation}\\
		\hline
		0.0031 & 0.4278
		\\
		\hline
	\end{tabular}
	\qquad\qquad\qquad
	\begin{tabular}{|c|c|}
		\hline
		\textbf{Mean P\&L}& \textbf{Standard Deviation}\\
		\hline
		-0.0009 & 0.2174
		\\
		\hline
	\end{tabular}
\end{center} 

\bigskip
\noindent Next, we have conducted a test whereby we carried out delta-hedging at every timestep, N=21 and N=84, and the charts above show the results of the delta hedge. As we can witness in the results, the expected value of the final P\&L seems to be close to each other but the standard deviation of hedging error when N=21 is almost twice of that of when N=84. This thus shows that there is a large variation in the P\&L when we hedge at a much lower frequency, which would be disastrous if spikes were to occur in the underlying prices between rebalancing trades. In addition, as can be seen from the charts above, the hedging errors exhibit negative skews (i.e long tail towards the left), whereby an investor can experience extreme left tail events (huge losses) which would be highly undesirable \\ \\

\section{Qualitative Assessment Of Results}

\noindent The assumption of the Black-Scholes replication strategy is to delta-hedge continuously till the option’s maturity but this is not made possible due to the real-world limitations. Therefore, that can only suggest discrete delta hedging in which one can short or long the underlying asset to remain delta-neutral at each time step. However, this still has its own pitfalls which will be discussed below. \\ \\

\noindent In the case of discrete delta hedging, we would only be sampling the underlying prices of the stock intermittently, i.e. at per time step of N = 21/84 steps, and the results would therefore not be a true indicator of the underlying volatility. Even with a known constant volatility and no spikes to the underlying asset price, the measured volatility across some paths would still deviate away from the 20\% constant volatility mark due to statistical fluctuations.  This therefore introduces replication error as the delta-hedge uses a constant volatility of 20\% that might not be the true delta to be hedged. This is highly undesirable as this leaves us to have a delta exposure which might lead to favour or disfavour to the P\&L.  In addition to that, we can also observe that the P\&L has a negative skewness which might expose us to severe downside risk due to our delta exposure. \\ \\

\noindent Therefore, in order to reduce potential downside risk due to the replication error and negative skewness, we can increase the delta-hedging frequency in order to minimize the standard deviation of our P\&L. As shown in the results, the standard deviation was almost halved once the delta hedge is performed at about four times per day as compared to about daily. Despite the existence of replication error still, increase in the hedging frequency (N) allows us to match much closely to the true volatility of the underlying price which can help reduce large undesirable losses in our trading P\&L. \\ \\

\noindent In conclusion, discrete delta hedging would lead to potential losses due to replication error, and one should look to increase the hedging frequency in order to reduce the potential downside risk. In addition to that, one should also incorporate other important factors into consideration such as the possibility of underlying prices spikes, changes in the volatility, and transaction costs, in order to create a much realistic delta hedging strategy. 

\end{document}